\documentclass{article}
\title{Information Retrieval Lab 2}
\author{Chiao-ting Fang \&\ Caroline Appleby}
\date{\today}
\begin{document}
\maketitle

\begin{enumerate}
\item \emph{What does the function extract_entity_names(t) do?  Explain.  Give a concrete example where you show input and output.}

The function expects an nltk Tree as argument, in the form that is returned by nltk.ne_chunk_sents(). In this Tree, named entities are marked by being a Tree with the label \texttt{NE}. From examination of the values of the \texttt{chunked_sentences} variable, Trees may also have the label \texttt{S}, conceptually sentence, or have no label. This corresponds to the output of the binary named entity chunker from NLTK. 

The function first checks that the tree has a label attribute, and that this attribute has some value.  If this is the case,  it next checks whether the value of this label is \texttt{NE} - conceptually whether this Tree is a named entity (rather than a sentence). If this is the case, it forms a string consisting of all the words in this subtree, and adds this string to the array \texttt{entity_names}. If the Tree in question is not a named entity, the function is called recursively on all the children of this Tree, checking if any of these are named entities.

The return value is a list of strings, where each string is a named entity found in the file given as input. 

%TODO example


\item \emph{Modify the script so that it sorts the recognized entities by frequency, instead of just returning a set of them.  Run the modified script again. How well do you think the script recognizes named entities?  Are there any false positives?  What are the ten most frequent entities and how frequent are they?}

The ten most frequent named entities are:
\begin{verbatim}
output goes here
\end{verbatim}


\item \emph{Consider the scenario where we want to recognize only named entities that represent locations (e.g., Antarctica or Lisbon).  Improve the script to incease precision and/or recall when recognizing locations.  You must make at least 2-3 separate improvements. Show the results of running your script, before and after, for each text, and use the results to argue for why your improvements are useful.  (You may want to test your script on more texts to ensure that it works in general and not just on the two sample books.)}


\end{enumerate}

\end{document}